Abudeif, A. M., Fat-Helbary, R. E., Mohammed, M. A., Alkhashab, H. M., \& Masoud, M. M. (2019). Geotechnical engineering evaluation of soil utilizing 2D multichannel analysis of surface waves (MASW)technique in New Akhmim city, Sohag, Upper Egypt. Journal of African Earth Sciences, 157, 103512. 

ASTM D4428/D4428M-14. (2014). Standard Test Methods for Crosshole Seismic Testing. ASTM International. 

ASTM D5753-18. (2018). Standard Guide for Planning and Conducting Geotechnical Borehole Geophysical Logging. ASTM International. 

ASTM D5777-18. (2018). Standard Guide for Using the Seismic Refraction Method for Subsurface Investigation. ASTM International. 

ASTM D6429-20. (2020). Standard Guide for Selecting Surface Geophysical Methods. ASTM International.

ASTM D7400/D7400M-19. (2019). Standard Test Methods for Downhole Seismic Testing. ASTM International. 

Cao, Y., Lu, Y., Zhang, Y., McDaniel, J. G., \& Wang, M. L. (2011). A fast inversion analysis algorithm for the spectral analysis of surface wave (SASW) method. Nondestructive Characterization for Composite Materials, Aerospace Engineering, Civil Infrastructure, and Homeland Security 2011, 7983(March), 79831E. 

Choon Byong Park, Richard D. Miller, \& Jianghai Xia. (1998). Imaging dispersion curves of surface waves on multi‐channel record. GEOPHYSICS , 1337–1380. 

Chris King, B. W., Witten, A. J., \& Reed, G. D. (1989). Detection and Imaging of Buried Wastes Using Seismic Wave Propagation. Journal of Environmental Engineering, 115(3), 23528-undefined. 

Deren Yuan, B., \& Nazarian, S. (1993). Automated Surface Wave Method: Inversion Technique. Journal of Geotechnical Engineering, 119(7), 1112–1126. 

Diersen, S., Lee, E. J., Spears, D., Chen, P., \& Wang, L. (2011). Classification of seismic windows using artificial neural networks. Procedia Computer Science, 4, 1572–1581. 

Dimililer, K., Dindar, H., \& Al-Turjman, F. (2021). Deep learning, machine learning and internet of things in geophysical engineering applications: An overview. Microprocessors and Microsystems, 80. 

Foti, S., Comina, C., Boiero, D., \& Socco, L. V. (2009). Non-uniqueness in surface-wave inversion and consequences on seismic site response analyses. Soil Dynamics and Earthquake Engineering, 29(6), 982–993. 

Foti, Sebastiano, Parolai, S., Albarello, D., \& Picozzi, M. (2011). Application of Surface-Wave Methods for Seismic Site Characterization. Surveys in Geophysics, 32(6), 777–825. doi.

García-Jerez, A., Piña-Flores, J., Sánchez-Sesma, F. J., Luzón, F., \& Perton, M. (2016). A computer code for forward calculation and inversion of the H/V spectral ratio under the diffuse field assumption. Computers and Geosciences, 97, 67–78. doi.org/10.1016/j.cageo.2016.06.016

George A. McMechan, \& Mathew J. Yedlin. (1981). Analysis of dispersive waves by wave field transformation. GEOPHYSICS , 46(6), 869–874. doi.org/10.1190/1.1441225

Guo, X. (2021). First-Arrival Picking for Microseismic Monitoring Based on Deep Learning. doi.org/10.1155/2021/5548346

Hayashi, K. (2012). Analysis of Surface-wave Data Including Higher Modes Using the Genetic Algorithm.
Kallivokas, L. F., Fathi, A., Kucukcoban, S., Stokoe, K. H., Bielak, J., \& Ghattas, O. (2013). Site characterization using full waveform inversion. Soil Dynamics and Earthquake Engineering, 47, 62–82. 

Leong, E. C., \& Aung, A. M. W. (2013). Global Inversion of Surface Wave Dispersion Curves Based on Improved Weighted Average Velocity Method. Journal of Geotechnical and Geoenvironmental Engineering, 139(12), 2156–2169. 

Lin, C.-P., Asce, M., Lin, C.-H., Dai, Y.-Z., \& Chien, C.-J. (2012). Assessment of Ground Improvement with Improved Columns by Surface Wave Testing. In P. E. Lawrence F. Johnsen, P. D. . C. E. Donald A. Bruce, \& P. E. . F. A. Michael J. Byle (Eds.), Proceedings of the Fourth International Conference on Grouting and Deep Mixing (pp. 483–492). 

Lin, C. H., Lin, C. P., Dai, Y. Z., \& Chien, C. J. (2017). Application of surface wave method in assessment of ground modification with improvement columns. Journal of Applied Geophysics, 142, 14–22. 

Lin, C. P., \& Chang, T. S. (2004). Multi-station analysis of surface wave dispersion. Soil Dynamics and Earthquake Engineering, 24(11), undefined. 

Lin, C. P., Lin, C. H., \& Chien, C. J. (2017). Dispersion analysis of surface wave testing – SASW vs. MASW. Journal of Applied Geophysics, 143, 223–230. 

Liu, M., Jervis, M., Li, W., \& Nivlet, P. (2020). Seismic facies classification using supervised convolutional neural networks and semisupervised generative adversarial networks. GEOPHYSICS, 85(4), O47–O58. 

Luke, B., Asce, M., \& Calderón-Macías, C. (2007). Inversion of Seismic Surface Wave Data to Resolve Complex Profiles. Journal of Geotechnical and Geoenvironmental Engineering, 133(2), 155–165. 

Luo, Y., Xia, J., Liu, J., Xu, Y., \& Liu, Q. (2009). Research on the middle-of-receiver-spread assumption of the MASW method. Soil Dynamics and Earthquake Engineering, 29(1), 71–79. 

Mahvelati, S., Asce, S. M., Coe, J. T., \& Asce, A. M. (2017). Multichannel Analysis of Surface Waves (MASW) Using Both Rayleigh and Love Waves to Characterize Site Conditions. Geotechnical Frontiers 2017: Transportation Facilities, Structures, and Site Investigation, 647–656. 

Matthews, M. C., Clayton, C. R. I., \& Own, Y. (2000). The use of field geophysical techniques to determine geotechnical stiffness parameters. Proceedings of the Institution of Civil Engineers - Geotechnical Engineering, 31–42. doi.org/10.1680/geng.2000.143.1.31

Moro, G. D. (2015a). Data Acquisition. In Surface Wave Analysis for Near Surface Applications (pp. 23–41). Elsevier. 

Moro, G. D. (2015b). Inversion and Joint Inversion. In Surface Wave Analysis for Near Surface Applications (pp. 87–102). Elsevier. 

Mousavi, S. M., Zhu, W., Sheng, Y., \& Beroza, G. C. (2019). CRED: A Deep Residual Network of Convolutional and Recurrent Units for Earthquake Signal Detection. Scientific Reports, 9(1). 

Olafsdottir, E. A., Bessason, B., \& Erlingsson, S. (2018). Combination of dispersion curves from MASW measurements. Soil Dynamics and Earthquake Engineering, 113, 473–487. 

Olafsdottir, E. A., Erlingsson, S., \& Bessason, B. (2018). Tool for analysis of multichannel analysis of surface waves (MASW) field data and evaluation of shear wave velocity profiles of soils. Canadian Geotechnical Journal, 55(2), 217–233.

Olafsdottir, E. A., Erlingsson, S., \& Bessason, B. (2020). Open-source masw inversion tool aimed at shear wave velocity profiling for soil site explorations. Geosciences (Switzerland), 10(8), 1–30. 

Osama AI-Hunaidi, M. (1994). Analysis of dispersed multi-mode signals of the SASW method using the multiple filter] crosscorrelation technique.

Park, C. B., Miller, R. D., \& Xia, J. (1999). Multichannel analysis of surface waves. GEOPHYSICS, 64(3), 659–992. 

Pelekis, P. C., \& Athanasopoulos, G. A. (2011). An overview of surface wave methods and a reliability study of a simplified inversion technique. Soil Dynamics and Earthquake Engineering, 31(12), 1654–1668. 

Penumadu, D., \& Park, C. B. (2005). Multichannel Analysis of Surface Wave (MASW) Method for Geotechnical Site Characterization. Earthquake Engineering and Soil Dynamics: Geo-Frontiers Congress 2005, 1–10. 

Peters, B., Haber, E., \& Granek, J. (2019). Neural networks for geophysicists and their application to seismic data interpretation. Leading Edge, 38(7), 534–540. 

Reichstein, M., Camps-Valls, G., Stevens, B., Jung, M., Denzler, J., Carvalhais, N., \& Prabhat. (2019). Deep learning and process understanding for data-driven Earth system science. Nature, 566(7743), 195–204. 

Ross, Z. E., Meier, M. A., \& Hauksson, E. (2018). P Wave Arrival Picking and First-Motion Polarity Determination With Deep Learning. Journal of Geophysical Research: Solid Earth, 123(6), 5120–5129. 

Roy, N., \& Jakka, R. S. (2017). Near-field effects on site characterization using MASW technique. Soil Dynamics and Earthquake Engineering, 97, 289–303. 

Roy, N., \& Jakka, R. S. (2018). Effect of data uncertainty and inversion non-uniqueness of surface wave tests on VS,30 estimation. Soil Dynamics and Earthquake Engineering, 113, 87–100. 

Subramaniam, P., Yunhuo, Z., Ng, Y. C. H., Danovan, W., \& Ku, T. (2020). Modal analysis of Rayleigh waves using classical MASW-MAM approach: Site investigation in a reclaimed land. Soil Dynamics and Earthquake Engineering, 128, 105902. 

Supranata, Y. E., Kalinski, M. E., \& Ye, Q. (2007). Improving the Uniqueness of Surface Wave Inversion Using Multiple-Mode Dispersion Data. International Journal of Geomechanics, 7(5). 

Taipodia, J., Dey, A., Gaj, S., \& Baglari, D. (2020). Quantification of the resolution of dispersion image in active MASW survey and automated extraction of dispersion curve. Computers and Geosciences, 135. 

Thorson, J. R., \& Claerbouts, J. F. (1985). Velocity-stack and slant-stack stochastic inversion. GEOPHYSICS, 50(12), 2727–2741. 

Tomeh, A. A., Alyateem, S., Malik, H., \& Malone, B. (2006). Geophysical Surveying \& Data Simulation Application to Geotechnical Investigations-A Cost Effective Approach for Developing Economical Foundation Engineering Design Criteria.

Tran, K. T., Mcvay, M., Faraone, M., \& Horhota, D. (2014). Detection of Embedded Anomalies Using 2-D Full Seismic Waveform Tomography. Geo-Congress 2014 Technical Papers: Geo-Characterization and Modeling for Sustainability, 2382–2393. 

Wang, Z., Sun, C., \& Wu, D. (2021). Automatic picking of multi-mode surface-wave dispersion curves based on machine learning clustering methods. Computers and Geosciences, 153, 104809. 

Wathelet, M., Jongmans, D., \& Ohrnberger, M. (2004). Surface-wave inversion using a direct search algorithm and its application to ambient vibration measurements. 

Wilkins, A. H., Strange, A., Duan, Y., \& Luo, X. (2020). Identifying microseismic events in a mining scenario using a convolutional neural network. Computers and Geosciences, 137, 104418. 

Williams, R. T., \& Penumadu, D. (2011). Multichannel Analysis of Surface Waves (MASW) at the National Geotechnical Engineering Site at Texas A\&M University (NGES/TAMU).

Xia, J., Miller, R. D., \& Park, C. B. (1999). Estimation of near-surface shear-wave velocity by inversion of Rayleigh waves. GEOPHYSICS, 64(3), 691–700. 

Xia, J., Xu, Y., \& Miller, R. D. (2007). Generating an image of dispersive energy by frequency decomposition and slant stacking. Pure and Applied Geophysics, 164(5), 941–956. 

Zhang, H., Ma, C., Pazzi, V., Li, T., \& Casagli, N. (2020). Deep Convolutional Neural Network for Microseismic Signal Detection and Classification. Pure and Applied Geophysics, 177(12), 5781–5797. 



 