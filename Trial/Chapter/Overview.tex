This thesis presents a strength improvement methodology for soft soil in Can Tho by
practical approach though laboratory experiments and Finite Element Method applied
ABAQUS 2D. On the one hand, the natural soil experiments in laboratory conducted to
obtain soft soil properties in Can Tho, and soil treated specimens by combining with
cement comprised of compressive and flexural samples are tested to investigate their
unconfined compressive strength and flexural strength. Particularly, both types of
specimens are made with different contents of cement from 100 to 500 kg per 1m3
natural soil (DCM100 to DCM500) and cured in different periods of time which are 07,
14 and 28 days. \\ [10pt]
On the other hand, deep cement mixing (DCM) piles supported embankment models are
established applied Modified Cam Clay model for soft soil and Mohr-Coulomb model
for DCM piles to investigate the bearing capacity, the effect of replacement area (AR)
and group coefficient (n) of DCM columns in soft soil. Although, this study presents
Can Tho soft soil properties, however, since the limitation of Can Tho soft soil
parameters fitted Modified Cam Clay model, the properties of Bangkok soft clay is used
to suit the Modified Cam Clay model. In general, because of the models’ establishment
purposes are the corporation between piles and soil, so the piles’ working mechanism in
soft soil is more considerable rather than the soil itself.
The natural soil experiments result that the soil in Can Tho is very soft with
approximately qu = 20 to 30 kPa with large strain,  strain = 5-14 percent observed from unconfined
compression test, and this strength increases very significant to qu = 1000 kPa at the
quatity of 300kg cement at 28 days (DCM300-28), with strain reduction to smaller
value,  strain = 0.7-2 percent. \\ [10pt]

Furthermore, the unconfined compressive strength development of soil cement
specimens is depended on cement content and also curing time. In terms of strength
development and cement content relationship, the unconfined compressive strength of
soil increases less significant in case of cement content is less than 250kg (DCM250).
In contrast, value of unconfined compressive strength develops very quickly in cases of cement content is from 300kg or higher quantity and the trend is similar regardless of
curing time. Considering the unconfined compressive strength develops with respect to
curing periods, from 7 to 14 days of curing, the strength increases slightly, however it
is grows strongly from the period from 14 to 28 days, qu = 1000 kPa (DCM300-28).
Looking at flexural strength, the flexural strength development is depended on cement
content and curing period, which is in a similar manner to in cases of compressive
specimens. The observation of strength undertaken by flexure test indicates that the
strength of flexural samples is around one-half compared to that of compressive
specimens. \\ [10pt]

In terms of FEM simulation, the results of bearing capacity of single pile (Qult = 510
kPa) is 11.6 percent lower than analytical approach (Qult = 577 kPa). The settlement below
pile tip is reduced very dramatically to less than 60 cm compared to controlled model
which has 1.2m of settlement. The pile failure at 1729 kPa, that leads to the conclusion
that the ultimate bearing capacity should be calculated by soil failure mechanism. The
multiple-pile patterns are generated to consider its ultimate bearing capacity under
increasing load, the outcomes illustrate that the increase of pile number with constant
length leads to the variation of ultimate bearing capacity and settlement. To be more
specific, the relationship between replacement area ratio (AR) and ultimate bearing
capacity (Qult) can be used to describe the simulation consequences. When the range of
AR is from 15 percent (Qult = 29248 kN for single pile method and Qult = 50560 kN for group
pile method) to 21.5 percent (Qult = 73205 kN for single pile method and Qult = 87570 kN for
group pile method), the bearing capacity increases associated with settlement reduction
from 1.73 to 1.19, for case 15 percent and 21.5 percent, respectively. Then when the AR value increased to be higher, the negative trend starts, it means that the settlement is higher
with lower ultimate bearing capacity. \\ [10pt]

With respect to group coefficient (n), which is also varied with the change in AR value.
Particularly, from 15 percent to 21 percent the value of group coefficient is higher than 1. However, when AR value reaches to 36.8 percent and 52 percent, the group coefficient reduces to be lower than 1.